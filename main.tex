% MUST use a4paper option
% MAY use twoside, smaller font, and other class - but not for Självständigt arbete i IT
\documentclass[a4paper,12pt]{article}
% Use UTF-8 encoding in input files
\usepackage[utf8]{inputenc}
% Use T1 font encoding to make the \hyphenation command work with UTF-8
\usepackage[T1]{fontenc}

% Om ni skriver på svenska, använd denna rad:
% \usepackage[english,swedish]{babel}
% If you are writing in English, use the following line INSTEAD of the previous (note order of parameters):
\usepackage[swedish,english]{babel}

% Use the template for thesis reports
\usepackage{UppsalaExjobb}

\usepackage{enumitem}
\usepackage{svg}
\usepackage{subcaption}
\usepackage{caption}
\usepackage{mwe}

% My packages
\usepackage{tabularx}
\usepackage{algorithm}
\usepackage{algorithmicx}
\usepackage{algpseudocode}
\usepackage{amsmath}
\usepackage{placeins}
\usepackage{afterpage}
\newcommand\emptypage{
    \null
    \thispagestyle{empty}
    \addtocounter{page}{-1}
    \newpage
    }

% För att göra ett index behövs
%  - \usepackage{makeidx}
%  - \makeindex i "preamble", dvs före \begin{document}
%  - \printindex, typiskt sist, före \end{document}
% - och att man lägger in \index{ord} på olika ställen i dokumentet
\usepackage{makeidx}
\makeindex


% Designval: per default används styckesindrag, men ibland blir det snyggare/mer lättläst med tomrad mellan stycken. Det åstadkoms av de följande raderna.
% Tycker ni om styckesindrag mera, kommentera bort nästa två rader.
\parskip=0.8em
\parindent=0mm

% Designval: vill ni ha en box runt figurer istället för strecken som är default, av-kommentera raden nedan. Obs att både \floatstyle och \restylefloat behövs.
%\floatstyle{boxed} \restylefloat{figure}

\begin{document}
% För att ställa in parametrar till IEEEtranS/IEEEtranSA behöver detta ligga här (före första \cite).
% Se se IEEEtran/bibtex/IEEEtran_bst_HOWTO.pdf, avsnitt VII, eller sista biten av IEEEtran/bibtex/IEEEexample.bib.
%%%% OBS: här ställer ni t.ex. in hur URLer ska beskrivas.
\bstctlcite{rapport:BSTcontrol}

% Set title, and subtitle if you have one
\title{Adapting and Evaluating Buddy Allocators for Use Within ZGC} % och uppsatsmetodik
% Use this if you have a subtitle
\subtitle{ZGC's New Best Friend}
%\subtitlesubtitle

% Set author names, separated by "\\ " (don't forget the space, or use newline)
% List authors alphabetically by LAST NAME (unless someone did significantly more/less, which should not be the case)
% For drafts, include your email addresses to make it easier to send peer reviews
\author{Casper Norrbin}

% Visa datum på svenska på förstasidan, även om ni skriver på engelska!
\date{\begin{otherlanguage}{swedish}  %\foreignlanguage doesn't seem to affect \today?
    Juli 2024
  \end{otherlanguage}}

% Använd detta om året för rapporten inte är innevarande år
%\year=2018

% Ange handledare, ämnesgranskare, examinator om dessa finns
\handledare{Erik Österlund}
% There is also \exthandledare
\reviewer{Tobias Wrigstad}
\examinator{Lars-Åke Nordén}

% This creates the title page
\maketitle
% \emptypage

% Change to frontmatter style (e.g. roman page numbers)
\frontmatter

%%%% OBS: Läs också källkoden till alla text/X.tex.
%%%% Tips: ni kan använda separata filer för de olika delarna i er rapport på motsvarande sätt,
%%%% men använd inte samma filnamn!

\begin{abstract}
  \input{text/abstract}
\end{abstract}

\begin{sammanfattning}
  \input{text/sammanfattning}
\end{sammanfattning}

% Innehållsförteckningen här.
\tableofcontents

% Här kan man också ha \listoffigures, \listoftables

\cleardoublepage

% Change to main matter style (arabic page numbers, reset page numbers)
\mainmatter

% Here comes the text of the report.

\section{Introduction}
\label{sec:introduction}
Effective memory management is a crucial for any software system. It can be categorized into either manual memory management, where the developer is responsible for both allocating and deallocating memory, or automatic memory management, where the system handles memory on behalf of the developer. Garbage collection, a type of automatic memory management, identifies and reclaims memory that is no longer in use. There are various different implementations of garbage collection that achieve this goal.

Java applications run within a Java Virtual Machine (JVM), and one such example is the Open Java Development Kit (OpenJDK). OpenJDK includes various garbage collectors, including the Z garbage collector (ZGC). ZGC organizes memory into many pages, which are operated on concurrently during garbage collection. Objects are allocated sequentially on these pages through a method known as bump-pointer allocation, where a pointer tracks the position of the most recently allocated object, increasing it, or ``bumping'' it, with each new allocation.

Because bump-pointer allocation always starts from the end address of the last allocation, it is unable to reuse previously allocated memory on the same page, that has since become free. To resolve this, ZGC either relocates all active objects to a new page, allowing the original page to be reset, or moves all objects one by one to the top of the page. An alternative to this method is employing a free-list-based allocator, which maintains a list of all unoccupied memory, thereby allowing allocations in any available space, independent of prior allocations.

Using free-lists in garbage collectors is nothing new. Concurrent Mark Sweep (CMS) is a garbage collector that relies on free-lists. It was part of OpenJDK until its deprecation and subsequent removal. CMS could perform allocations without the constraints imposed by bump-pointer allocation methods, thanks to its use of free-lists. It is clear that free-lists offer certain advantages, but they may not always be the optimal choice for every situation. Using a free-list-based allocator within an existing garbage collector can offer significant advantages, as the garbage collector has more information available about the objects it allocates that the allocator can use. This thesis investigates the potential integration of a free-list-based allocator within ZGC, with focus on examining possible adaptations that can boost allocator efficiency within a garbage collection context, without actual implementation in ZGC. 

%%% Local Variables:
%%% mode: latex
%%% TeX-master: "main"
%%% End:


\newpage
\subsection{Purpose and Goals}
\label{sec:purpose}
The purpose of this thesis is to investigate how it is possible to adapt an existing free-list-based allocator for use within ZGC, and to evaluate the effects of these adaptations on allocator performance. The goal is to leverage insights of this work to direct subsequent efforts in the domain of free-list-based allocators in garbage collection.


The research questions this thesis seeks to answer are the following:
\begin{enumerate}
  \item What changes are neccesary when adapting an allocator for use within ZGC?
  \item Is it possible to leverage knowledge of integration into ZGC to improve allocator performance?
\end{enumerate}

%%% Local Variables:
%%% mode: latex
%%% TeX-master: "main"
%%% End:


\vspace{-0.4cm}
\subsection{Delimitations}
\label{sec:delimitations}
The primary delimitation of this thesis is the exclusion of allocator integration within ZGC. The focus is on examining the allocators and their adaptations independently. Pages in ZGC can be of three types: small, medium, or large. The modifications made focus solely on small pages, as these are the most common. However, insights gained from small pages may guide potential adaptations for medium and possibly large pages.

%%% Local Variables:
%%% mode: latex
%%% TeX-master: "main"
%%% End:


\subsection{Individual Contributions}
\label{sec:individual_contrubitons}
\input{text/individual_contributions}

\subsection{Outline}
\label{sec:outline}
The rest of this thesis is organized as follows: Section~\ref{sec:introduction} outlines the purpose, goals, delimitations, and individual contributions. Section~\ref{sec:background} provides context on memory management, fragmentation, memory allocation strategies, and ZGC. Section~\ref{sec:relatedwork} reviews existing memory allocators and the use of free-lists in garbage collection. Section~\ref{sec:buddy} details the binary buddy allocator and its allocation, deallocation, and state determination methods. Section~\ref{sec:method} describes the implementation process, from reference design to evaluation. Section~\ref{sec:buddyadditions} introduces enhancements such as the binary tree and inverse buddy allocators. Section~\ref{sec:adaptations} discusses specific adaptations of the allocators for ZGC. Section~\ref{sec:evaluation} outlines the performance and fragmentation testing approaches. Section~\ref{sec:results} presents the findings on page overhead, performance, and fragmentation. Section~\ref{sec:discussion} interprets these results, and Section~\ref{sec:futurework} suggests further research directions. Finally, Section~\ref{sec:conclusions} summarizes key insights and their implications.

% \paragraph{Acknowledgement}
% \input{text/acknowledgement}

\newpage
\section{Background}
\label{sec:background}
\input{text/background}

\subsection{Memory Management}
\label{sec:memory_management}
\input{text/memory_management}

\subsection{Fragmentation}
\label{sec:fragmentation}
\input{text/fragmentation}

\subsection{Memory Allocation}
\label{sec:memory_allocation}
\input{text/memory_allocation}

\subsection{Garbage Collection}
\label{sec:gc}
\input{text/GC}

\subsection{OpenJDK}
\label{sec:openjdk}
\input{text/openjdk}

\subsection{The Z Garbage Collector}
\label{sec:zgc}
\input{text/zgc}

\subsubsection{Pages}
\label{sec:zpage}
\input{text/zpage}

\newpage
\section{Related Work}
\label{sec:relatedwork}
Given the large number of existing allocator designs, one might conclude that most problems with dynamic memory allocation have already been solved. In most cases, a general-purpose allocator designed for use in any system or environment performs well on average in terms of response time and fragmentation. However, there may be areas for improvement depending on how much the use case can be narrowed down. For example, a garbage collector has access to more information about the objects being allocated than most other users of an allocator.

\subsection{Memory Allocators}

One of the first memory allocation techniques is the simple buddy allocator, conceived by Knowlton \cite{buddy}. The buddy allocator partitions memory into blocks, each sized as a power of two. Initially, the entire memory is set as a single block. In the first allocation, this block is divided recursively into halves until the block size closest to the allocation size is obtained. These new differently-sized blocks can then be utilized for subsequent allocations. The two resulting blocks from the splitting of a block are called ``buddies''. When a block is deallocated and its buddy is also unallocated, the two blocks are combined into a larger block. This merging process continues recursively as long as the buddy is also unallocated.

The simple design of the buddy allocator provides ample room for enhancements. Peterson and Norman presented a more generalized version of the buddy allocator~\cite{genbuddy}, allowing block sizes that are not powers of two and allowing buddies of unequal sizes. Recent advances have also included making the allocator lock-free~\cite{nbbs} and enabling the same block to exist in multiple sizes simultaneously to improve the efficiency of smaller allocations~\cite{park2014ibuddy}.

The buddy allocator has seen wide use in real environments and applications. A notable example is its use within the Linux kernel for allocating physical memory pages~\cite{linuxbuddy}. In addition, the buddy allocator can be combined with various other allocation techniques to improve efficiency. For example, \texttt{jemalloc}~\cite{jemalloc}, created for the FreeBSD operating system, is a general-purpose allocator that incorporates multiple strategies, including the buddy allocator, to achieve overall efficiency.

\subsection{Free-Lists in Garbage Collection}

Using free-lists in garbage collection systems is not a new concept. For example, the basic mark-sweep algorithm, as outlined in Section~\ref{sec:gc}, necessitates the use of a free-list. The static position of live objects can lead to significant fragmentation. Consequently, simpler allocation methods, such as bump-pointer allocation, become impractical. Instead, employing a strategy based on free-lists for allocation proves to be a more effective approach to managing and allocating memory surrounding live objects.

Among the free-list strategies outlined in Section~\ref{sec:freelist_allocation}, the segregated-fit method is particularly noted for its effectiveness~\cite{gchandbook}. Programs that assume garbage collection generally allocate more than those that manually handle memory. Therefore, the significance of segregated-fit free-lists becomes particularly crucial as program allocations increase.

The Concurrent Mark Sweep (CMS) garbage collector~\cite{cms}, introduced in JDK 1.4, is a concrete example of free-lists being used within garbage collectors. CMS implements a concurrent variant of the mark-sweep algorithm, utilizing free-lists for managing memory. Although it was deprecated in JDK 9, CMS demonstrates the practicality of using free-lists in garbage collectors. This highlights their versatility and adaptability across different garbage collection algorithms and implementations.

There is potential for free-list-based allocators to exist within already existing garbage collection systems. Using multiple allocation strategies in tandem can enhance allocation efficiency, as can be seen in various allocators~\cite{jemalloc}. With a free-list-based allocator integrated into a garbage collection system, the system can leverage the benefits of both strategies to improve performance. For example, integrating a buddy allocator into ZGC would allow for fast allocation using bump-pointer allocation, while also providing the benefits of free-lists for managing memory surrounding live objects in situations where bump-pointer allocation is not as efficient.

%%% Local Variables:
%%% mode: latex
%%% TeX-master: "main"
%%% End:


\newpage
\section{Binary Buddy Allocator}
\label{sec:buddy}
The concept of a buddy memory allocator was first introduced by Knowlton~\cite{buddy}. This section explains how the original buddy memory allocator operates, as envisioned by Knowlton. The core idea of the buddy allocator involves dividing memory into blocks and pairing them as ``buddies''. Buddy pairs are split when allocating memory and merged when deallocating memory to allow allocations of different sizes. The simple design of the buddy allocator makes it attractive for various applications. For example, the Linux kernel implements a modified binary buddy allocator to manage physical memory~\cite{linuxbuddy}, and \texttt{jemalloc} adopts a buddy allocator for one of its memory allocation methods~\cite{jemalloc}.

\subsection{Allocation and Deallocation}
The binary buddy allocator is the original and most basic type of buddy allocator. It partitions memory into blocks with sizes that are powers of 2. A free list of available blocks of different sizes is maintained to keep track of the free blocks. Initially, all memory is one large block. Figure~\ref{fig:buddystart} shows the initial state of a binary buddy allocator with a total memory of 128 bytes. It can be seen that the entire memory is contained within a single block, which is also included in the free list.

\begin{figure}[h]
    \centering
    \includesvg[width=0.61\textwidth]{figures/bbuddy_initial.svg}
    \caption{The initial state of the free-list and memory in a binary buddy allocator. Shown are the memory and the free-list of available blocks.}
    \label{fig:buddystart}
\end{figure}

When requesting memory, a block of the smallest power-of-two size that can accommodate the requested memory will be returned. If there are no blocks of that size available, a larger block will be recursively split into two until a block of the correct size is available. The split blocks are added to the free list for future allocations. An algorithmic explanation of this process can be found in Algorithm~\ref{alg:bbuddy_alloc}.

\begin{algorithm}[h]
    \caption{Binary buddy allocation algorithm}
    \label{alg:bbuddy_alloc}
    \begin{algorithmic}[1]
        \Statex \textbf{Procedure} Allocate(size)
        \State $level \gets \text{smallest power of 2} \geq size$

        \Statex \textbf{Find a suitable block}
        \State $block \gets \text{no\_block}$
        \For{$i \gets level$ to \text{max\_level}}
        \If{\text{free list for level $i$ is not empty}}
        \State $block \gets \text{remove first block from free list of level } i$
        \State \textbf{break}
        \EndIf
        \EndFor

        \If{$block = \text{no\_block}$}
        \State \Return \text{allocation\_failed}
        \EndIf

        \Statex \textbf{Split the block if necessary}
        \While{$\text{level of } block > level$}
        \State $block, buddy \gets \text{split } block \text{ into two buddy blocks}$
        \State \text{add } $buddy$ \text{ to its corresponding free list}
        \EndWhile

        \State \Return $block$
    \end{algorithmic}
\end{algorithm}

\newpage
As an example, consider Figure~\ref{fig:buddysplit}, which shows the resulting state of the previous figure after requesting 16 bytes of memory. The initial 128-byte block has been split multiple times to fulfill the request, with the leftmost block being returned and the addition of various block sizes to the free list.

\begin{figure}[h]
    \centering
    \includesvg[width=0.62\textwidth]{figures/bbuddy_allocated.svg}
    \caption{The state of the free-list and memory in a binary buddy allocator after one 16-byte allocation. Shown are the memory and the free-list of available blocks. Each new line represents one step in the allocation process.}
    \label{fig:buddysplit}
\end{figure}

\FloatBarrier

Deallocation is the reverse process of allocation. When a block is deallocated, it is merged with its buddy if the buddy is also deallocated. Merging creates a new, larger block. After merging, the original block and its buddy are removed from the free list, and the new, larger block is inserted. The larger block checks the status of its buddy, and the merging process continues recursively until merging is no longer possible. An algorithmic explanation of this process can be found in Algorithm~\ref{alg:bbuddy_dealloc}. To demonstrate this process, consider the situation where the 16-byte block in Figure~\ref{fig:buddysplit} is deallocated. In this scenario, the block would go through a recursive merging process with its buddies until it reaches the initial state shown in Figure~\ref{fig:buddystart}.

\begin{algorithm}[h]
    \caption{Binary buddy deallocation algorithm}
    \label{alg:bbuddy_dealloc}
    \begin{algorithmic}[1]
        \Statex \textbf{Procedure} Deallocate(block)
        \State $level \gets \text{level of } block$

        \While{true}
        \State $buddy \gets \text{find buddy of } block$
        \If{$\text{buddy is not free}$}
        \State $\text{add } block \text{ to free list of level } level$
        \State \textbf{break}
        \Else
        \State $\text{remove } buddy \text{ from free list of level } level$
        \State $block \gets \text{merge } block \text{ and } buddy$
        \State $level \gets level + 1$
        \EndIf
        \EndWhile
    \end{algorithmic}
\end{algorithm}

\FloatBarrier
\subsection{Determining Buddy State}
When deallocating a block, the allocator must determine the state of its corresponding buddy. This is done using a bitmap, with each entry indicating if a block has been allocated or not. The allocator assigns a unique number to each block that can potentially be allocated. Figure~\ref{fig:buddyorder} illustrates all possible blocks and their respective numbering. The bitmap requires a total of $2^n - 1$ entries, where $n$ is the number of levels of potential block sizes.

\begin{figure}[h]
    \centering
    \includesvg[width=0.37\textwidth]{figures/bbuddy_order.svg}
    \caption{Unique numbering of each possible block in a binary buddy allocator.}
    \label{fig:buddyorder}
\end{figure}
% \vspace{-0.7cm}
The bitmap is correlated with the available blocks in the free-list and is modified both when splitting and merging blocks. For example, Figure~\ref{fig:buddybmapallocated} shows the state of the bitmap of a binary buddy allocator after one 16-byte allocation. It can be seen that the largest block has been split down to the smallest size and that the buddies of the used blocks are marked in the bitmap as available.
% \vspace{-0.1cm}

\begin{figure}[H]
    \centering
    \includesvg[width=0.635\textwidth]{figures/bbuddy_bmap_allocated.svg}
    \caption{The state of the bitmap and free-list of a binary buddy allocator after one 16-byte allocation.}
    \label{fig:buddybmapallocated}
\end{figure}

%%% Local Variables:
%%% mode: latex
%%% TeX-master: "main"
%%% End:


\clearpage
\section{Method}
\label{sec:method}
The methodology was divided into five distinct phases:

\begin{enumerate}
    \item \textbf{Implement the Reference Design:} Implement the original binary buddy allocator to establish a foundation for further development.
    \item \textbf{Investigate Key Aspects:} Identify important aspects of using a memory allocator within a garbage collection context, particularly ZGC, through research and experimentation with source code.
    \item \textbf{Make Modifications Based on Existing Research:} Modify the reference allocator based on prior research to enhance its performance.
    \item \textbf{Adapt for Garbage Collection:} Further modify the allocator for specific use cases within garbage collection.
    \item \textbf{Evaluate Performance:} Measure memory fragmentation and allocation speed for both the reference allocator and the modified versions.
\end{enumerate}

The initial phase involved implementing the original binary buddy allocator designed by Knowlton \cite{buddy}. This implementation provided a deeper understanding of its functionality and established a solid foundation for subsequent development. The base allocator was tested using both unit tests and real-world applications. By exporting the allocator as a shared library within a wrapper, it was possible to test real-world programs by overriding the malloc/free operations using the \texttt{LD\_PRELOAD} environment variable.

Once the base allocator was implemented, the goal of phase two was to understand the nuances of memory allocation within the context of garbage collection and ZGC. This involved conducting a literature review, examining specific parts of the ZGC source code, and performing experiments. This phase was crucial for grasping the complexities of ZGC and its memory allocation mechanisms, identifying potential areas for modification, and determining which changes would have the greatest impact. The changes made and the reasons behind them can be found in Sections~\ref{sec:buddyadditions} and~\ref{sec:adaptations}.

With a thorough understanding of the binary buddy allocator and ZGC, the third phase involved integrating modifications inspired by prior research. The most significant changes were applied, sometimes requiring the implementation of conflicting concepts, which were then compared and evaluated. The focus was on leveraging existing work to enhance overall performance, rather than to start from scratch. These adjustments were not exclusively tailored for garbage collection but aimed to improve general performance or produce positive results when used in garbage collection processes.

In the fourth phase, improvements were implemented that were specifically relevant to the allocator's use in a garbage collection environment. These modifications were derived from insights and knowledge acquired in the preceding phases, rather than based on existing research. The goal was to introduce features tailored for specific scenarios within ZGC without compromising performance in other aspects of ZGC or elsewhere.

The implementation process was incremental, with each minor modification being tested and validated before proceeding to the next step. As the allocator evolved, the number of unit tests increased, which ensured comprehensive test coverage of its capabilities. Additionally, the allocator was tested with preexisting programs after each modification to ensure stability and performance.

The final phase involved evaluating the reference allocator and the modifications. Several key metrics were considered during the evaluation, including fragmentation, memory usage, and allocation speed. These metrics were essential for assessing the allocator's performance and comparing different versions. Different metrics may hold greater significance in different scenarios, while others may have different requirements. A detailed explanation of the evaluation process can be found in Section~\ref{sec:evaluation}.

%%% Local Variables:
%%% mode: latex
%%% TeX-master: "main"
%%% End:


\newpage
\section{Improved Buddy Allocators}
\label{sec:buddyadditions}
As of the time of writing, the original binary buddy allocator is nearly 60 years old. Over the years, numerous improvements and advances have been made, including refining the initial design~\cite{genbuddy} and adapting it for new applications and scenarios~\cite{nbbs, park2014ibuddy}. This section covers two notable adaptations of the binary buddy allocator: the binary tree buddy allocator and the inverse buddy allocator. These adaptations aim to improve the performance of the original buddy allocator in specific scenarios, such as when allocating small blocks or when minimizing block splitting and merging.

\subsection{Binary Tree Buddy Allocator}
The original binary buddy allocator keeps track of available blocks by organizing them into different free-lists. Consequently, every time an allocation or deallocation occurs, entries must be added to and removed from these lists. If a block needs to be split or can be merged, more operations are needed. To avoid these costly free-list manipulations, an implicit free-list can be implemented using a binary tree structure. This tree includes all the information in the original free-list.

An example of such an implementation is that of Restioson~\cite{btbuddy}. In this implementation, the tree stores the maximum allocation size for each block or its children. Since blocks are evenly divided into two, only the block's ``level'' needs to be stored to determine its size. The initial state of a binary tree buddy allocator is illustrated in Figure~\ref{fig:btbuddyinitial}. Each level of the tree represents a specific size, and the root level corresponds to the entire memory space.

\begin{figure}[h]
  \centering
  \includesvg[width=0.73\textwidth]{figures/btbuddyinitial.svg}
  \caption{The initial state of the binary tree and its free-list equivalent in a binary tree buddy allocator.}
  \label{fig:btbuddyinitial}
\end{figure}

\subsubsection{Allocation in a Binary Tree Buddy Allocator}
When allocating with a binary tree, the level required for the allocation size is first determined. Then, a binary search is started from the root node of the tree. The node is checked to see if the available level is high enough; if it is, it checks the left child. If the left child does not have a large enough block available, the right child is chosen instead. This search process continues recursively on the child node until the correct level within the tree is reached. Upon finding a suitable node, its level is set to 0, and the parent nodes are updated to reflect the new state of their children. An algorithmic explanation of this process can be found in Algorithm~\ref{alg:btbuddy_alloc}.

\begin{algorithm}
  \caption{Binary tree allocation algorithm}
  \label{alg:btbuddy_alloc}
  \begin{algorithmic}[1]
    \Statex \textbf{Procedure} Allocate($size$)
    \State $level \gets \text{smallest power of 2} \geq size$

    \State $node \gets \text{root of tree}$
    \Statex \textbf{Find an available node}
    \While{$\text{level of } node > level$}
    \State $left, right \gets \text{get children of } node$
    \If{$\text{level avaliable in } left \geq level$}
    \State $node \gets left$
    \Else
    \State $node \gets right$
    \EndIf
    \EndWhile

    \If{$\text{level avaliable in } node < level$}
    \State \Return \text{allocation\_failed}
    \EndIf

    \State $\text{mark } node \text{ as unavaliable}$
    \Statex \textbf{Update the tree}
    \ForAll{$parent \text{ nodes starting from } node$}
    \State $\text{level available in } parent \gets \text{max(}left,right\text{)}$
    \EndFor
    \State \Return $node$
  \end{algorithmic}
\end{algorithm}

To illustrate this, Figure~\ref{fig:btbudydallocated} shows the state of a binary tree buddy allocator following the allocation of a 32-byte and a 16-byte block. It can be seen that the initial allocation is placed in the leftmost node at the second-lowest level, while the subsequent allocation is placed to the right of it at the lowest level. Consequently, the entire left branch of the binary tree now only contains one 16-byte block, as indicated by one being stored in the respective nodes.

\begin{figure}[h]
  \centering
  \includesvg[width=0.73\textwidth]{figures/btbuddyallocated.svg}
  \caption{The state of the binary tree and its free-list equivalent in a binary tree buddy allocator after one 32-byte and one 16-byte allocation.}
  \label{fig:btbudydallocated}
\end{figure}

\subsubsection{Deallocation in a Binary Tree Buddy Allocator}
Deallocation in a binary tree buddy allocator is also a simple process. Initially, the node that has been deallocated is set to its corresponding level, followed by updating its parent nodes to reflect this change. When the buddy of a block is also free, signified by it having the same level as the recently deallocated node, they are merged. The merging operation in a binary tree involves increasing the level of the parent by one, indicating that the entire memory space it represents is now available. An algorithmic explanation of this process can be found in Algorithm~\ref{alg:btbuddy_dealloc}.

\begin{algorithm}
  \caption{Binary tree deallocation algorithm}
  \label{alg:btbuddy_dealloc}
  \begin{algorithmic}[1]
    \Statex \textbf{Procedure} Deallocate($block$)
    \State $level \gets \text{level of } block$
    \Statex \textbf{Set the level of the node}
    \State $node \gets level$

    \Statex \textbf{Update the tree}
    \ForAll{$parent \text{ nodes starting from } node$}
    \If{$left = right \text{ and } left \text{ is free completely}$}
    \State $\text{level available in } parent \gets left + 1$
    \Else
    \State $\text{level available in } parent \gets \text{max(}left,right\text{)}$
    \EndIf
    \EndFor
  \end{algorithmic}
\end{algorithm}

\subsection{Inverse Buddy Allocator}
Allocating the smallest block size in a binary buddy allocator requires the most total block splits. To improve the speed of allocating these blocks, adjustments are necessary. Park et al.~\cite{park2014ibuddy} proposed an alternative method for organizing the metadata of a binary buddy allocator, which they call an inverse buddy allocator (iBuddy allocator). An iBuddy allocator can allocate individual blocks in constant time, although at the expense of slower allocations for larger blocks. Rather than dividing larger blocks into smaller ones to fulfill allocations, all blocks are initially split, and merging is only done during larger allocations.

\subsubsection{Allocation in an Inverse Buddy Allocator}
In an iBuddy allocator, blocks of different sizes that overlap the same memory are all present in the free-list and bitmap at the same time. In the iBuddy allocator, the bitmap takes on additional meaning. When a block is marked as free, it implies that all smaller blocks at that memory location are also free and available for allocation. This enables the possibility of allocating a smaller size in a larger block, and keeping the rest of the space in that block available for future allocations. Consider Figure~\ref{fig:ibuddyinitial}, which shows a valid initial state of an iBuddy allocator. It can be seen that all potential addresses for the smallest-size blocks are present in the free-list but at different levels. For example, the first block is available at the uppermost level, while the second block is available at the lowest level.

\begin{figure}[h]
  \centering
  \includesvg[width=0.702\textwidth]{figures/ibuddy_bmap_initial.svg}
  \caption{The initial state of the free-list and bitmap in an inverse binary buddy allocator.}
  \label{fig:ibuddyinitial}
\end{figure}

Due to the new structure of the bitmap, it is possible to allocate single-sized blocks in place of larger blocks without affecting other blocks in the bitmap. When using an iBuddy allocator, the largest available block in the free-list is always utilized for allocation, irrespective of the allocation size. Consider Figure~\ref{fig:ibuddyallocated}, which shows an allocation of one block of the smallest size. The largest block from the preceding figure has been taken out of the free-list and bitmap, whereas the remainder of the free-list and bitmap remains unaltered. This allocation did not influence other blocks, and all other smallest-size blocks are still accessible at different levels. For the next smallest-size allocation, the 64-byte block would be used. No explicit block splitting is needed, and by storing the highest level of the free-list with available blocks, the allocation of smallest-sized blocks can be done in constant time.

\begin{figure}[h]
  \centering
  \includesvg[width=0.702\textwidth]{figures/ibuddy_bmap_allocated.svg}
  \caption{The state of the free-list and bitmap in an inverse binary buddy allocator after one
    16-byte allocation.}
  \label{fig:ibuddyallocated}
\end{figure}

Fast allocation speed for small blocks comes at the cost of slow allocation speed for larger blocks. In contrast to a binary buddy allocator, blocks cannot simply be removed, as there are now overlapping smaller blocks present in the free-list and bitmap. These must be removed to avoid allocating the same memory location twice. An algorithmic explanation of this process can be found in Algorithm~\ref{alg:ibuddy_alloc}.

Consider Figure~\ref{fig:ibuddyallocated2}, which shows the same allocator as in the previous figure, now with an additional 64-byte allocation. It can once again be seen that the highest-level block has been removed, but now, all the blocks under it have also been cleared from both the free-list and the bitmap. For larger blocks, the allocation speed is proportional to the number of smallest-sized blocks needed to fill the requested size.

\begin{algorithm}[h]
  \caption{iBuddy allocation algorithm}
  \label{alg:ibuddy_alloc}
  \begin{algorithmic}[1]
    \Statex \textbf{Procedure} Allocate($size$)
    \State $level \gets \text{smallest power of 2} \geq size$
    \State $block \gets \text{remove the first block from the highest-level non-empty free list}$

    \If{$level = 0$}
    \State \Return $block$
    \ElsIf{$\text{level of } block < level$}
    \State \Return \text{allocation\_failed}
    \EndIf

    \Statex \textbf{Remove all lower blocks}
    \For{$i \gets level$ to \text{0}}
    \ForAll{$\text{possible blocks in level } i$}
    \State $\text{remove } block \text{ from free list at level } i$
    \EndFor
    \EndFor

    \State \Return $block$
  \end{algorithmic}
\end{algorithm}

\begin{figure}[h]
  \centering
  \includesvg[width=0.635\textwidth]{figures/ibuddy_bmap_allocated2.svg}
  \caption{The state of the free-list and bitmap in an inverse binary buddy allocator after one 16-byte allocation and one 64-byte allocation.}
  \label{fig:ibuddyallocated2}
\end{figure}
    
\FloatBarrier
\subsubsection{Deallocation in an Inverse Buddy Allocator}
In an inverse buddy allocator, no blocks are explicitly merged during deallocation; instead, the level at which they are placed back into the free-list indicates the size of the available block. During deallocation, the status of the block's buddy is checked; if it is free, instead of merging, the deallocated block rises one level. This process repeats until a buddy is no longer free, after which the block is inserted at that layer. An algorithmic explanation of this process can be found in Algorithm~\ref{alg:ibuddy_dealloc}.

As an example of the deallocation process, consider Figure~\ref{fig:ibuddydeallocated}, which shows the previous figure, now after deallocating the first 16-byte allocation. The buddy is first checked at the bottom level, then again at the level above that, to finally end up at the third level, where it is inserted and marked in the bitmap as free. The complexity of deallocating single blocks is the same as that of the binary buddy allocator, but less work needs to be done, as no explicit merging is done, only checks for the blocks' buddies.

\begin{figure}[h]
  \centering
  \includesvg[width=0.635\textwidth]{figures/ibuddy_bmap_deallocated.svg}
  \caption{The state of the free-list and bitmap in an inverse binary buddy allocator after one
    16-byte allocation, one 64-byte allocation, and one 16-byte deallocation.}
  \label{fig:ibuddydeallocated}
\end{figure}

Similar to allocation, the deallocation speedup for smaller blocks comes at the cost of larger blocks. When deallocating a large block, all the smaller blocks within that block also need to be inserted back into the free-list and bitmap. Thus, the operation of deallocating a large block is equivalent in cost to individually deallocating each of the smallest-size blocks that make up the larger block. This is more costly than a binary buddy, where larger blocks require the least work as they merge the fewest number of times.

\begin{algorithm}[H]
  \caption{iBuddy deallocation algorithm}
  \label{alg:ibuddy_dealloc}
  \begin{algorithmic}[1]
    \Statex \textbf{Procedure} Deallocate($block$)
    \ForAll{$\text{lowest-level blocks } current\_block \text{ that fit in } block$}
    \State $level \gets 0$
    \State $buddy \gets \text{find buddy of } current\_block \text{ at } level$
    \While{$buddy \text{ is free}$}
    \State $level \gets level + 1$
    \State $buddy \gets \text{find buddy of } current\_block \text{ at } level$
    \EndWhile
    \State $\text{add } current\_block \text{ to free list at level } level$
    \EndFor

  \end{algorithmic}
\end{algorithm}

\clearpage
\section{Adapting the Buddy Allocator for use With ZGC}
\label{sec:adaptations}
As discussed in Section~\ref{sec:buddyadditions}, there is ample room to improve the original binary buddy allocator. The two designs discussed have different strengths, so both are implemented to compare and contrast with each other and the original design. The binary tree allocator closely resembles the binary buddy allocator in its allocation method, whereas the iBuddy allocator differs significantly. Both allocators are investigated to determine the most efficient allocation and deallocation processes for all possible allocatable sizes.

Further enhancements can be developed that are orthogonal to the specific allocator implementations. These enhancements can be general optimizations or specifically tailored to the context of operating within ZGC. Constraining the use case enables the allocators to assume access to additional information or to narrow their functionality. This section discusses the adaptations made to the allocators to improve their performance within ZGC.

\subsection{Allocator Configurability} \label{sec:adaptationsconfig}
There are numerous scenarios in which a garbage collector could benefit from an advanced memory allocator, many of which are beyond the scope of this thesis. Each scenario has distinct characteristics and unique requirements. Improvements should enhance specific scenarios without causing adverse effects in other situations.
If a modification improves one use case but negatively impacts another, it should be possible to disable that change. This flexibility allows each scenario to utilize the most suitable configuration for that specific case by enabling or disabling the desired features. The adaptations made are indepentant of one another, allowing for easy configuration changes of specific features.

Making the allocator configurable improves its applicability to a wide range of scenarios. Although this versatility is advantageous, it may lead to limitations in certain instances. Prioritizing configurability can hinder the allocator from being fully optimized for one particular use case. Significant modifications to enhance one specific scenario could adversely affect others. The choice was made to prioritize configurability instead of restricting the allocator to a single use case. This decision allows the allocator to be used in a variety of scenarios, with the user selecting the most suitable configuration for their specific needs.

\subsection{Allocator Metadata Implementations} \label{sec:adaptationsmetadata}
For some scenarios, a specific metadata implementation may be more beneficial than others. The metadata structures can be altered to either prioritize memory efficiency or speed, depending on needs. This section covers the choices made regarding the metadata storage of the allocators.

\subsubsection{Page Metadata} \label{sec:metadataexpl}

The design of the buddy allocator requires storing additional metadata separate from the data on the page. This metadata, which stores the status of possible blocks, incurs a fixed overhead for each page allocated, rather than increasing with each allocation. The size of the metadata is influenced by the total number of possible blocks rather than their size. Consequently, having larger block sizes lead less of a proportional overhead, whereas smaller block sizes result in significantly higher proportional overhead.

The basic binary buddy allocator is the most efficient in terms of memory usage, requiring the least data. The state of each potential block must be stored, which can be represented by a single bit. This data is only used during deallocation, so the states of two buddy blocks can be merged into one bit using XOR operations, effectively halving the memory requirement. This optimization is not possible for the iBuddy allocator, as blocks can be deallocated without knowing the state of their buddy.

% \subsubsection{Binary Tree Metadata} \label{sec:binarytreeexpl}
The binary tree allocator retains the design of the original binary buddy allocator but stores its metadata in a different format. This approach allows for faster allocation and deallocation, as there is no explicit free-list that needs to be managed.

The binary tree is stored as a flattened byte array, with each level stored contiguously following the preceding one. Most levels have each byte representing the value of a single block, but the lowest levels deviate from this pattern. Each node in the tree holds the maximum possible level that can be allocated within it or its children. The smallest blocks can only store $1$ or $0$, enabling memory optimization by compacting and storing 8 blocks within each byte in the array. Optimizing this level, which contains half of the total blocks, results in memory savings of $43.75$\%. Similarly, the data for the next two levels can be compacted to 2 bits per block, further reducing memory by $28.13$\%. Higher levels require at least $4$ bits of information and constitute a smaller fraction of total blocks, so their memory is not optimized to avoid performance impacts from bit operations.

\subsubsection{Finding Block Buddies} \label{sec:findbuddiesexpl}
When deallocating, the allocator must determine the correct block to deallocate based on the memory address. Since an address can belong to any block level, additional information is required. The simplest solution is to require the user to provide the allocated size, allowing the allocator to calculate the deallocation level. This method is fast and incurs minimal overhead as the allocator does not need to store additional data. However, it requires the user to keep track of the allocated size, which may not always be feasible.

Alternatively, the level of each allocation can be stored inside the block or in a separate data structure. Although this is simple and allows for fast lookups, it is not very memory-efficient. If the data is stored within a block, the usable space decreases. Storing data separately for each possible block location results in a slightly larger but constant overhead.

A third approach uses a bitmap to track which blocks have been split. This method is very memory-efficient but requires more operations to find the correct blocks. Figure~\ref{fig:buddybmapsplit} illustrates this bitmap after a 16-byte block allocation, showing all blocks above the allocated block marked as split. The smallest blocks do not require an entry in the bitmap, as they cannot be split. During deallocation, the allocator starts at the top of the bitmap and traverses down until the block is no longer marked as split.

\begin{figure}[h]
    \centering
    \includesvg[width=0.64\textwidth]{figures/bbuddy_bmap_split.svg}
    \caption{The state of the split blocks-bitmap and free-list of a binary buddy allocator after one 16-byte allocation.}
    \label{fig:buddybmapsplit}
\end{figure}

All the three options are implemented and can be configured in the allocators, with the most suitable option depending on the specific usage scenario. For general use, the bitmap approach is the most memory-efficient across various allocator configurations.

For use within ZGC, data structures responsible for tracking size can be omitted, delegating this function to the garbage collector. The collector has sufficient data from its live analysis and object headers to compute the size of each object and allocation. This approach eliminates additional overhead, making optimal use of existing resources.

\subsection{Using the Allocators on Already Allocated Memory} \label{sec:freerangeexpl}
In certain situations, using an advanced allocator to manage memory may not be the most efficient option. For instance, bump-pointer allocation could be used initially, switching to a more advanced allocator when significant external fragmentation arises. This approach maximizes the speed advantage of bump-pointer allocations while leveraging advanced allocators when necessary.

In the context of ZGC, the allocator must be able to allocate around memory already in use. This imposes limitations on the allocator, as it no longer ``owns'' the memory it uses for allocation. The allocator instead has to allocate in holes of available memory. This affects where the storage of allocator metadata, as it cannot be stored within its own memory region.

When working with memory not fully controlled by the allocator, the allocator starts with no available memory for allocation. The user must indicate the locations of occupied memory or the intervals between them, defining free regions that the allocator can use. In ZGC, live analysis of a page provides information on live objects, enabling deallocation of the intervals between them.

To implement functionality for this scenario, the allocator's initial state is fully occupied with the smallest-sized blocks, avoiding accidental merging of blocks overlapping with occupied regions. During the deallocation of a specific range, the largest fitting blocks are added to the free-list, and each block and its split children are marked as no longer split. Figure~\ref{fig:deallocrange} illustrates this process, showing blocks added to the free-list covering the entire free range.

\begin{figure}[h]
    \centering
    \includesvg[width=0.62\textwidth]{figures/deallocrange.svg}
    \caption{The blocks added to the free-list after deallocating the range up until the last block.}
    \label{fig:deallocrange}
\end{figure}

\subsection{Lazy Splitting and Merging of Blocks} \label{sec:lazyexpl}
Splitting and merging blocks is costly and should be avoided if possible. The binary buddy allocator merges blocks whenever possible, often requiring immediate splitting for subsequent allocations. Lee and Barkley \cite{lazylayer} designed a modified buddy allocator that delays block merging. As long as the allocation size distribution remains consistent, costs related to splitting and merging can be avoided. A simpler version of Lee and Barkley's design is implemented, with a second free-list (lazy layer) on top of the buddy allocator (buddy layer). This separate free-list does not perform any splitting or merging and solely stores blocks of different sizes.

When deallocating using a lazy layer, the block is inserted into the lazy layer's free-list. If the lazy layer reaches a certain threshold of blocks, the block is inserted into the buddy layer as normal. During allocation, the lazy layer is checked first for a suitable block, followed by splitting blocks from the buddy layer if necessary. If both steps fail, the lazy layer can be emptied to allow smaller blocks to be merged to meet the required allocation size.

It is logical for different block sizes to have different thresholds, as the frequency of sizes differs. Since most allocations are small, it is also logical that smaller blocks should have a higher threshold. Storing large blocks in the lazy layer would also reserve these large blocks for only that size, which could lead to smaller allocations not being able to be fulfilled. When implementing this, the smallest block size has the highest threshold, with each level above that halving the threshold of the previous level. The default threshold is set to 1000. A high value is desired, as ZGC can often deallocate many objects to then allocate many objects of the same size. The lazy layer needs the capacity to support this, which is why the threshold is set relatively high.

Different block sizes have different thresholds, with the smallest block size having the highest threshold. Large blocks in the lazy layer could prevent smaller allocations from being fulfilled, so thresholds decrease for larger block sizes. The default threshold is set to 1000 for the smallest block size, with the threshold halving for each level above that. A high threshold is desired, as ZGC could deallocate many objects to afterward allocate many objects of the same size. The lazy layer needs the capacity to support this, which is why the threshold is set relatively high.

\subsection{Allocator Regions} \label{sec:concurrencyexpl}
In the original binary buddy allocator, the largest block equals the entire provided memory, which is then split into smaller blocks during allocation. This approach can degrade performance in multithreaded programs with concurrent allocations, as only one allocation can split a large block at a time.

To avoid this, the maximum block size is set lower than the entire memory size, dividing the memory into smaller regions, each with a normal buddy block structure. Park et al.~\cite{park2014ibuddy} suggest this for the iBuddy allocator. For example, splitting memory into two regions, each with a maximum block size of half the total memory, allows for two concurrent allocations. If one region is prioritized, fragmentation would be reduced as one region would fulfill smaller allocations, leaving space for larger allocations in the second region.

In ZGC, the maximum allocation size in a small page is $256$ KiB, a fraction of the page size of 2 MiB. This allows for eight allocations of the maximum size within a page. The allocator can be divided into eight regions, each covering one-eighth of a page. Fewer regions would limit the number of concurrent allocations, while more regions would restrict the maximum allocation size.

One allocator controlls all regions, keeping shared metadata such as the lazy layer, while each region has its own buddy structure. This design allows for concurrent allocations in different regions, as long as the allocations are evenly distributed. Scalability extends up to the number of regions, with additional allocations needing to wait for prior allocations to complete.

Situations could arise where some regions become more heavily utilized than others, decreasing throughput. When a thread initiates an allocation, it is assigned a specific region. If another thread is allocating in that region, it checks the subsequent regions. If all regions are in use, the thread waits. Upon entering a region, the necessary allocation logic is performed. If an allocation fails due to insufficient space, the thread exits the region and retries in the next one. Allocation requests fail when all regions cannot accommodate the request.

\subsection{Combining and Configuring the Adaptations} \label{sec:adaptationsall}
All the mentioned adaptations build on top of any of the three different buddy allocators discussed in Sections~\ref{sec:buddy} and \ref{sec:buddyadditions}. Each adaptation can be enabled or disabled independently, allowing for a wide range of configurations. The user can select the most suitable configuration for their specific use case, optimizing the allocator for their needs. While the adapatations are designed to enhance the allocator's performance within ZGC, the configurability allows the user to optimize the allocator for any use case.

To configure the allocator, the user first selects which base allocator to use. The binary buddy allocator is the most memory-efficient, the binary tree allocator is a faster alternative at the cost of memory, and the iBuddy allocator is designed for fast small block allocations. The user then configures the adaptations: selecting the minimum and maximum block sizes, the number of regions, the metadata storage method, and the lazy layer thresholds. The allocator is then ready for use, optimized for the user's specific needs.


% \section{Implementing Allocator Adaptations}
% \label{sec:adaptationsimpl}
% \input{text/adaptationsimpl}

\newpage
\section{Evaluation Methodology}
\label{sec:evaluation}
Performance, memory usage, and fragmentation of the allocators are critical metrics that were evaluated in this study. These are important factors for a memory allocator, and improvements in one often affect others. This section outlines the evaluation process used to assess the performance and memory usage of the allocators, as well as the fragmentation they introduce. Performance was measured by the time required for allocation and deallocation requests. Memory usage refers to the additional memory utilized by the allocator for its operations, and fragmentation refers to the inefficiencies in memory utilization resulting from the allocation strategy.

As previously stated, a major limitation of this project was that the allocator was not integrated into ZGC, even though this was its intended purpose. Such an integration would have provided the simplest and most intuitive method of testing the allocator, as this is the environment in which the allocator would be used. The challenge of not doing this was to identify other benchmarks that could indicate the allocator's performance within ZGC.

\subsection{Allocator Configurations}
The three versions of the buddy allocator---the binary buddy allocator, the binary tree allocator, and the iBuddy allocator---were tested and compared against each other. They all share the same base configuration of block sizes and regions, with a minimum block size of $16$ bytes, a maximum block size of $256$ KiB, and a total of $8$ regions.

\subsection{Performance}
Performance is influenced by both the type of allocator used and the context of its use. Different allocators perform differently depending on the task; therefore, their performance varies across different programs. Instead of relying on a single program to compare all allocators, the time required for the allocation and deallocation operations was measured for each allocator. These measurements were used to compare the allocators' performance across different block sizes and allocation patterns.

\subsubsection{Measuring Performance} \label{sec:perfmeas}
The time required for a single allocation or deallocation for each block size was observed to provide insight into the performance of each allocator. Additionally, timing the allocation of a constant memory size using various block sizes demonstrated the performance of repeated allocations and deallocations.

For the single allocation and deallocation benchmarks, time was measured for every possible block size ranging from $2^4$ to $2^{18}$. These figures match the potential block sizes in a ZGC small page. To minimize noise, the average of $100 000$ allocations was calculated for each size and each allocator. The duration was recorded before and after the allocation call using the POSIX function \texttt{clock\_gettime()} with the \texttt{CLOCK\_MONOTONIC\_RAW} clock.

For the page-fill benchmark, a contiguous 2 MiB memory block was filled with every possible block size ranging from $2^4$ to $2^{18}$, matching ZGC's allocation sizes and page size. For each allocator and block size, the memory was filled $10 000$ times. The duration was recorded using the POSIX function \texttt{clock\_gettime()} with the \texttt{CLOCK\_MONOTONIC\_RAW} clock, starting before the initial allocation and ending after the final one.

These tests showed different performance aspects of the different allocators for various allocation sizes, which were used to draw several conclusions. To estimate the performance of a particular program, one would examine its allocation distribution and compare it with the test results to determine the most suitable allocator for that specific application.

\subsubsection{System Specifications}
Performance evaluations were performed using two different machines, detailed in Table~\ref{table:performancespecs}. Each memory allocator was compiled using GCC 11.4.0, adhering to the C++14 standard (\texttt{-std=c++14}) and using optimization level two (\texttt{-O2}).

\begin{table}[h]
    \begin{tabular}{lll}
        \textbf{Configuration}    & \textbf{Machine A}                                       & \textbf{Machine B}   \\ \hline
        CPU                       & Intel® Core™ i7-1270P                                    & AMD Opteron™ 6282 SE \\ \hline
        Sockets / Cores / Threads & 1 / 4P 8E / 16                                           & 2 / 16 / 32          \\ \hline
        Frequency (Base / Turbo)  & 2.2 GHz / 4.8 GHz                                        & 2.6 GHz / 3.3 GHz    \\ \hline
        L1 Cache                  & 448 KiB                                                  & 512 KiB              \\ \hline
        L2 Cache                  & 9 MiB                                                    & 32 MiB               \\ \hline
        L3 Cache                  & 18 MiB                                                   & 24 MiB               \\ \hline
        Memory                    & 16 GiB                                                   & 126 GiB              \\ \hline
        OS                        & \multicolumn{2}{l}{Ubuntu 22.04.4 LTS (Jammy Jellyfish)}                        \\ \hline
        Kernel                    & 6.5.0-17-generic                                         & 5.15.0-101-generic   \\
    \end{tabular}
    \centering
    \caption{Machines used for performance benchmarks.}
    \label{table:performancespecs}
\end{table}

\newpage
\subsection{Fragmentation} \label{sec:frageval}
\subsubsection{Internal Fragmentation}
Internal fragmentation occurs when allocation sizes are rounded to the nearest power of two to align with block sizes, and it is identical in all the allocators. This results in unused memory for every allocation that is not a perfect power of two. To quantify this, two counters were used: one for the requested allocation size and another for the total allocated size. Each allocation operation incremented these counters, while deallocation operations decremented them. At any point during program execution, these two counters could be compared to measure the amount of internal fragmentation. The percentage of internal fragmentation, or wasted memory, was calculated as:

$\text{Internal fragmentation} = \frac{\text{Used memory - Requested memory}}{\text{Used memory}}$

\subsubsection{External Fragmentation}
External fragmentation occurs when allocated blocks are placed in memory in a way that inhibits larger allocations. Even if the total space available is greater than the requested size, there may be no single contiguous block large enough to satisfy that request. The different allocators have different policies about where blocks are placed, so fragmentation differs between them.

\subsubsection{Measuring Fragmentation}
Fragmentation is highly dependent on a specific pattern of allocation and deallocation. A program that consecutively only allocates powers of two would experience no internal or external fragmentation. However, this scenario is unrealistic, as programs typically mix allocations and deallocations across a broad range of sizes. To illustrate potential fragmentation levels in a program, a simulation was performed that used the modified allocators in a way that created high fragmentation. This simulation provided a rough estimate of the maximum fragmentation that could arise from typical use of the allocator.

This test used two modules: one for creating a distribution of allocation sizes and another for producing a sequence of allocations and deallocations based on that distribution. Using these, it was possible to measure both internal and external fragmentation at any point within the sequence of allocations.

The chosen allocation distribution was based on the observation that most allocations in Java programs are small, with a few large allocations and occasionally very large ones. Samples were drawn from a Poisson distribution with a mean of $\lambda = 6$. The samples were converted to allocation sizes by using them as exponents with base 2, and a variance of $\pm 50\%$ was applied to the converted samples to introduce spread.

To make the evaluation targeted towards ZGC, the values were aligned to $8$, and any values smaller than $2^4$ or larger than $2^{18}$ were excluded to adhere to the limited allocation size range for small pages. This resulted in a distribution centered on $2^6$, with most values clustering near this point and a gradual but steep decrease in the frequency of higher values.

The sequence of allocations and deallocations was created to introduce significant fragmentation without using a deliberate pattern. First, a sequence based on the allocation size distribution was generated. Following this sequence, allocations were made until a failure occured. Subsequently, half of the allocations by size were randomly deallocated, resulting in fragmentation throughout the entire memory. This process was iterated numerous times to ensure extensive fragmentation.

External fragmentation was measured each time an allocation attempt was unsuccessful, providing insight into the allocator’s ability to manage fragmented memory. Internal fragmentation was measured simultaneously, capturing the impact of random size variations on memory utilization. This approach ensured a thorough evaluation of the allocator's performance under conditions of high fragmentation.

%%% Local Variables:
%%% mode: latex
%%% TeX-master: "main"
%%% End:


\newpage
\section{Results}
\label{sec:results}

\subsection{Memory Overhead}
The binary buddy allocator is the most efficient in terms of memory usage, requiring the least data. As discussed in Section~\ref{sec:metadataexpl}, the allocator only needs to store the state of each block, which can be represented by a single bit. Additonally, the binary buddy allocator can merge the states of two buddy blocks into a single bit using XOR operations. In the ZGC configuration, this resulted in a total overhead of $16$ KiB per $2$ MiB page, or $0.78$\%.

The iBuddy allocator also requires storing each block's state, but the same optimizations are not possible due. Each block needs to set and read its state individually, as operations can be executed on multiple blocks at once. In the ZGC configuration, this led to an overall overhead of $32$ KiB per $2$ MiB page, or $1.56$\%.

The binary tree allocator requires additional information to be stored. Storing only the state of each block is insufficient, as each block needs to maintain a record of the highest level accessible below it. However, significant optimizations were applied to the lower layers, as described in Section~\ref{sec:metadataexpl}, reducing the total overhead to $57$ KiB per $2$ MiB page, or $2.78$\%.

% \vspace{-0.2cm}
\subsection{Performance}
The results of the single allocation benchmark defined in Section~\ref{sec:perfmeas} are presented in Figure~\ref{fig:allocbenchmark}. Figures \ref{fig:allocA} and \ref{fig:allocB} show the results for machines A and B, respectively. The performance of the allocators was similar across both machines, differing by a constant speed factor.

Both the binary buddy and binary tree allocators demonstrated faster allocation of larger block sizes, with the latter showing superior performance for smaller block sizes. The iBuddy allocator was faster at allocating smaller blocks but had reduced speed for larger blocks. Utilizing the lazy layer significantly improved performance across all block sizes, with the greatest impact at smaller sizes.

\begin{figure}[h]
  \centering
  \begin{subfigure}{\textwidth}
    \centering
    \captionsetup{justification=centering}
    \includesvg[width=1.02\linewidth]{figures/alloc_laptop.svg}
    \caption{Allocation benchmark results from machine A.}
    \label{fig:allocA}
  \end{subfigure}
  \vspace{-0.5cm}
  \rule{\textwidth}{0.1pt}
  \begin{subfigure}{\textwidth}
    \centering
    \captionsetup{justification=centering}
    \includesvg[width=1.02\linewidth]{figures/alloc_server.svg}
    \caption{Allocation benchmark results from machine B.}
    \label{fig:allocB}
  \end{subfigure}
  \caption{Performance of individual allocations across various block sizes and allocators. The benchmark is run on both machines, and each bar displays the mean results from 100 000 iterations.}
  \label{fig:allocbenchmark}
\end{figure}

\FloatBarrier

The results of the single deallocation benchmark defined in Section~\ref{sec:perfmeas} are presented in Figure~\ref{fig:deallocbenchmark}. Figures \ref{fig:deallocA} and \ref{fig:deallocB} show the results for machines A and B, respectively. The performance of the allocators was similar across both machines, differing by a constant speed factor.

Both the binary buddy and binary tree allocators demonstrated faster deallocation of larger block sizes, with the latter showing superior performance for smaller block sizes, although less than during allocation. The iBuddy allocator was faster at deallocating smaller blocks but had reduced speed for larger blocks. Utilizing the lazy layer significantly improved performance across all block sizes, with the greatest impact at smaller sizes.

\begin{figure}[h]
  \centering
  \begin{subfigure}{\textwidth}
    \centering
    \captionsetup{justification=centering}
    \includesvg[width=1.02\linewidth]{figures/dealloc_laptop.svg}
    \caption{Deallocation benchmark results from machine A}
    \label{fig:deallocA}
  \end{subfigure}
  \rule{\textwidth}{0.1pt}
  \begin{subfigure}{\textwidth}
    \centering
    \captionsetup{justification=centering}
    \includesvg[width=1.02\linewidth]{figures/dealloc_server.svg}
    \caption{Deallocation benchmark results from machine B}
    \label{fig:deallocB}
  \end{subfigure}
  \caption{Performance of individual allocations across various block sizes and allocators. The benchmark is run on both machines, and each bar displays the mean results from 100 000 iterations.}
  \label{fig:deallocbenchmark}
\end{figure}

\FloatBarrier

The results of the page-fill benchmark defined in Section~\ref{sec:perfmeas} are presented in Figure~\ref{fig:allocpage}. Figures \ref{fig:allocpageA} and \ref{fig:allocpageB} show the results for machines A and B, respectively. The performance of the allocators was similar across both machines, differing by a constant speed factor.

The results illustrate the characteristics of each allocator when making multiple allocations. The binary buddy and binary tree allocators scale linearly with block size due to their improved allocation speed for larger blocks, with the binary tree allocator being the faster of the two.
The iBuddy allocator was fastest for the smallest block sizes, but its inverse scaling makes it only marginally faster when allocating larger blocks, quickly flattening out as block size increases.

\begin{figure}[h]
  \centering
  \begin{subfigure}{0.496\textwidth}
    \centering
    \captionsetup{justification=centering}
    \includesvg[width=1\linewidth]{figures/page_laptop.svg}
    \caption{Allocation results from machine A}
    \label{fig:allocpageA}
  \end{subfigure}
  \begin{subfigure}{0.496\textwidth}
    \centering
    \captionsetup{justification=centering}
    \includesvg[width=1\linewidth]{figures/page_server.svg}
    \caption{Allocation results from machine B}
    \label{fig:allocpageB}
  \end{subfigure}
  \caption{Time taken to allocate 2 MiB across various block sizes and allocator versions. The benchmark is run on both machines, and each point displays the mean results from 1 000 iterations.}
  \label{fig:allocpage}
\end{figure}

\FloatBarrier
\subsection{Fragmentation}
\subsubsection{Internal Fragmentation}
Internal fragmentation remains consistent across all allocator versions, as they round block sizes equally and use the same allocation pattern. Table~\ref{table:fraginternal} presents measures that quantify the internal fragmentation created by the allocators. Although fragmentation was considerable, the standard deviation is low, indicating consistent fragmentation with a low spread. The chosen allocation pattern greatly influenced fragmentation levels, causing nearly all allocations to require rounding up.

\begin{table}[h]
  \begin{tabular}{|l|l|}
    \hline
    \textbf{Measure}    & \textbf{Value} \\ \hline
    Minimum:            & 24.4\%         \\ \hline
    Maximum:            & 40.9\%         \\ \hline
    Mean:               & 32.0\%         \\ \hline
    Median:             & 32.3\%         \\ \hline
    Standard Deviation: & 2.1\%          \\ \hline
  \end{tabular}
  \centering
  \caption{Measurements of internal fragmentation based on 1,000 observations. All allocator versions experience the same internal fragmentation.}
  \label{table:fraginternal}
\end{table}

\subsubsection{External Fragmentation}
External fragmentation manifested itself differently in each allocator version due to their different strategies for placing allocated blocks. Figure~\ref{fig:fragext} illustrate the distribution of free blocks in terms of number and total size for the binary buddy, binary tree, and iBuddy allocators, respectively. The total free space is less important than the distribution of free blocks across different block sizes.

The binary buddy allocator showed the largest number of free blocks around $2^7$, but these constituted a minor portion of the overall free space. Most of the space was occupied by blocks around $2^{14}$, suggesting that allocations of this size were possible, and most unsuccessful allocations exceeded this size.

The binary tree allocator behaved similar to the binary buddy allocator, with the largest number of blocks around $2^7$.
Although it had more smaller-sized blocks, these made up only a small fraction of the total free space and could be attributed to a different allocation order.

Most of the space was occupied by blocks around $2^{13}$, which is less than the binary buddy allocator. This suggests that the binary tree allocator was better at utilizing memory, as it had fewer large blocks. The overall free space was also less than that of the binary buddy allocator, indicating that the binary tree allocator was more efficient in using memory than the binary buddy allocator.

The iBuddy allocator stands out due to its aggressive splitting policy. The most common size of free blocks was $2^6$, which also constituted a substantial portion of the total free space. The largest total space was found around $2^{13}$, although it was more evenly distributed between sizes. This concentration at $2^6$ resulted from most allocations exceeding this size, which led to blocks being split until this size. The overall free space was greater than that of the other two allocators, suggesting less efficiency in utilizing memory.

% \begin{figure}[h]
%   \centering
%   \includesvg[width=1\textwidth]{figures/frag_binary.svg}
%   \caption{Mean external fragmentation over 1 000 observations of the binary buddy allocator when allocating and deallocating randomly.}
%   \label{fig:fragextbinary}
% \end{figure}

% \begin{figure}[h]
%   \centering
%   \includesvg[width=1\textwidth]{figures/frag_bt.svg}
%   \caption{Mean external fragmentation over 1 000 observations of the binary tree allocator when allocating and deallocating randomly.}
%   \label{fig:fragextbt}
% \end{figure}

% \begin{figure}[h]
%   \centering
%   \includesvg[width=1\textwidth]{figures/frag_ibuddy.svg}
%   \caption{Mean external fragmentation over 1 000 observations of the iBuddy allocator when allocating and deallocating randomly.}
%   \label{fig:fragextibuddy}
% \end{figure}

\begin{figure}[h]
  \centering
  \begin{subfigure}{\textwidth}
    \centering
    \captionsetup{justification=centering}
    \includesvg[width=1\linewidth]{figures/frag_binary.svg}
    \caption{Mean external fragmentation of the binary buddy allocator.}
    \label{fig:fragextbinary}
  \end{subfigure}
  \rule{\textwidth}{0.1pt}
  \begin{subfigure}{\textwidth}
    \centering
    \captionsetup{justification=centering}
    \includesvg[width=1\linewidth]{figures/frag_bt.svg}
    \caption{Mean external fragmentation of the binary tree allocator.}
    \label{fig:fragextbt}
  \end{subfigure}
  \rule{\textwidth}{0.1pt}
  \begin{subfigure}{\textwidth}
    \centering
    \captionsetup{justification=centering}
    \includesvg[width=1\linewidth]{figures/frag_ibuddy.svg}
    \caption{Mean external fragmentation of the iBuddy allocator.}
    \label{fig:fragextibuddy}
  \end{subfigure}
  \caption{Mean external fragmentation over 1 000 observations of the three allocators when allocating and deallocating randomly.}
  \label{fig:fragext}
\end{figure}

\FloatBarrier
%%% Local Variables:
%%% mode: latex
%%% TeX-master: "main"
%%% End:


\newpage
\section{Discussion}
\label{sec:discussion}
The evaluation of the adapted allocators presents considerable challenges. Because these allocators are not integrated into ZGC, definitive conclusions regarding their performance are difficult to reach. Only proxy benchmarks are possible, serving as indicators of potential performance. Isolating the allocators for testing removes the influence of ZGC or other user-defined logic. This isolation can result in, for example, cache-locality effects that do not accurately reflect real-world scenarios. However, these assessments enable a comparison of the allocators on a uniformly defined basis and allow for a discussion of their comparative effectiveness.

Performance and external fragmentation benchmarks demonstrate that the binary tree allocator consistently outperforms the binary buddy allocator, but at the cost of increased memory overhead. Furthermore, the iBuddy allocator exhibits significant differences in allocation performance and external fragmentation distribution compared to the other two allocators. The choice of an allocator for a particular scenario should depend on the most critical performance metric for that use case.

Internal fragmentation is the most significant issue faced by all the allocators. The requirement to round up block sizes leads to considerable waste, especially with larger allocations. Since most allocations do not align with a perfect power of two, fragmentation occurs with each allocation, diminishing the efficiency of any buddy allocator in utilizing memory space. In the context of ZGC, this problem can be mitigated by strategically grouping object allocations to minimize internal fragmentation. Further discussion of this strategy is found in Section~\ref{sec:futureworkZ} as future work.

External fragmentation can only be measured empirically, which is why the allocation/deallocation pattern is chosen to induce significant fragmentation without maximizing it. Under this pattern, all allocator versions experience high external fragmentation, particularly at larger block sizes, where most of the total space is located. This suggests that failures predominantly occur with very large allocation requests and that a wide range of allocation sizes presents a major weakness, increasing the likelihood of failures for large requests. Limiting the range of sizes would likely decrease external fragmentation. Consequently, within ZGC, it should be presumed that additional allocatable space exists even if an allocation attempt fails. Assuming that memory is fully utilized could lead to considerable memory wastage.

\newpage
The unique behavior of the iBuddy allocator is evident in the benchmarks. Notably, its performance for large block allocations is significantly worse compared with small allocations in the other two allocators. Its only advantage is a slight increase in speed at the smallest allocation sizes. However, its strategy of aggressively splitting blocks quickly inhibits the allocation of larger blocks. Therefore, the iBuddy allocator is only feasible when nearly all allocations are among the smallest sizes. If a considerable number of allocations exceed this size, there will be a notable decrease in both performance and usable memory.

Considering all factors, the binary tree allocator emerges as the most promising option for implementation in ZGC among the three discussed. It consistently outperforms the binary buddy allocator in all benchmarks, while the iBuddy allocator is impractical due to the reasons discussed. Since memory consumption is not a critical concern, the additional memory expense of the binary tree allocator is justified by its speed improvement. Although the binary tree allocator is slower at smaller allocation sizes, this drawback can be mitigated by using the lazy layer, keeping good performance across all allocation sizes. This configuration is deemed the most promising for integrating a buddy allocator within ZGC.

However, the intrinsic weaknesses of buddy allocators remain. Internal fragmentation will be significant, even if it is possible to reduce it. Blocks still need to be aligned to their size, which could become problematic if previous memory is awkwardly placed. In addition, the cost of frequent allocator initialization could become non-negligible. These issues should be taken into account when integrating with ZGC, as they could affect both performance and memory efficiency. Future work should focus on strategies to mitigate these weaknesses and explore alternative allocation algorithms that may offer improved performance and reduced fragmentation in the context of ZGC.

%%% Local Variables:
%%% mode: latex
%%% TeX-master: "main"
%%% End:


\newpage
\section{Future Work}
\label{sec:futurework}
\subsection{Integration Into ZGC} \label{sec:futureworkZ}

The most significant future step is to integrate the allocators into ZGC. As highlighted in Section~\ref{sec:individual_contrubitons}, Gärds~\cite{niclas} has concurrently to this thesis explored the integration of an allocator into ZGC, focusing on a free-list-based allocator to enhance the efficiency of relocating objects within ZGC. Future efforts could build on Gärds' work by incorporating a buddy allocator into ZGC or exploring alternative integration scenarios not considered by Gärds.

When integrating a buddy allocator specifically, the garbage collector can implement strategies to mitigate some of its drawbacks. The primary issue is the considerable internal fragmentation during allocation. To minimize this, ZGC could allocate multiple objects simultaneously to align more closely with size requirements or use fragmented memory from an allocation for future use. This approach requires that objects allocated together be deallocated collectively. However, as detailed in Section~\ref{sec:freerangeexpl}, this is not an issue when deallocating ranges.

\subsection{Smaller Minimum Allocation Size} \label{sec:futureworkLiliput}
Currently, the minimum allocation size in ZGC is 16 bytes, constrained by the header size of Java objects. Efforts are underway to decrease the size of these headers, potentially lowering the smallest object size to 8 bytes \cite{liliput}. Since this development is ongoing, it was not considered in this thesis.

Reducing the minimum allocation size would introduce a new bottom layer for the allocators. Practically, this results in memory overhead per page doubling as it becomes necessary to manage and track smaller allocations. Additionally, for the binary buddy and iBuddy allocators, storing the two pointers needed for the free-list would no longer be feasible. These pointers would need to be transformed into offsets from the beginning of the page. Although the size of the pages makes this adjustment manageable, it would still introduce some performance overhead.

\newpage
\subsection{Further Allocation Optimizations} \label{sec:futureworkOptimizations}
Improving the binary tree storage format in the binary tree allocator might be feasible. Currently, the levels of the tree are stored sequentially, requiring progressively larger jumps when iterating over the levels. To increase spatial locality, the larger tree could be broken down into several subtrees stored consecutively. This would place the segments closer together, reducing the likelihood of cache misses and potentially increasing performance.

Currently, all the adapted allocators use the same concurrency approach, which involves dividing memory into regions and only permitting concurrent allocations in different regions. However, since their designs differ, it would be worth investigating different designs for each allocator. For example, a lock-free mechanism could achieve concurrency for the binary tree allocator. As the binary tree allocator navigates its tree structure, compare-and-swap (CAS) atomic operations could replace the current subtraction operations. Conflicts could arise with other threads, but threads can back off to resolve these conflicts. Consequently, it is advisable to keep distributing allocations evenly using regions, but with the locking code removed, thereby only allocating concurrently within the same region when necessary.

% alla samma, dåligt
% undersöka anpassing för specifika versioner
% binary tree extra bra!!! därför att......
% 

\subsection{Combining Allocation Strategies} \label{sec:futureworkCombine}
Another area of future research could involve exploring hybrid approaches that combine the strengths of various buddy allocator designs. This concept parallels the strategies used in other allocators like the linux kernel~\cite{linuxbuddy} and \texttt{jemalloc}~\cite{jemalloc}, which combine multiple allocation techniques to optimize performance and memory utilization.

For instance, a hybrid buddy allocator could employ a binary tree allocator for large block sizes and an iBuddy allocator for small block sizes, thereby leveraging the advantages of both approaches. These allocators could be distributed across different memory regions, optimizing their performance based on specific memory requirements. Alternatively, one could investigate methods to enable these allocators to work in tandem on the same memory region, dynamically switching between allocation strategies as needed to maximize efficiency and minimize fragmentation.


\newpage
\section{Conclusions}
\label{sec:conclusions}
This thesis has explored the adaptation and evaluation of different buddy allocators for integration within the Z Garbage Collector (ZGC). Through analysis and benchmarking, the binary tree allocator emerged as the most promising candidate, consistently outperforming the traditional binary buddy allocator in all key benchmarks. Although the iBuddy allocator demonstrated faster allocation speeds for smaller blocks, its poor performance with larger block allocations and significant internal fragmentation makes it impractical for uses where larger allocations occur regularly.

By tailoring the allocator to the specific needs of ZGC, several changes have been implemented to enhance efficiency and introduce additional features. Using already-available data allowed for further optimizations. Notably, a novel approach to initializing and employing the allocator was implemented: it begins fully occupied, with the user identifying which sections are empty and available for allocation. These modifications are designed to facilitate the seamless integration of the allocator into ZGC in the future.

To fully understand how the allocator would perform in ZGC, additional research is needed. So far, only proxy benchmarks have been utilized, showing promise for potential real-world applications. Therefore, future efforts should focus on developing strategies to mitigate the fundamental shortcomings of buddy allocators in terms of internal fragmentation and alignment. Additionally, integrating the allocator within ZGC would enable extensive testing and would be crucial in refining the allocator's performance.

%%% Local Variables:
%%% mode: latex
%%% TeX-master: "main"
%%% End:


\newpage

%%%% Referenser - SE OCKÅ APPENDIX

% Use one of these:
%   IEEEtranS gives numbered references like [42] sorted by author,
%   IEEEtranSA gives ``alpha''-style references like [Lam81] (also sorted by author)
\bibliographystyle{IEEEtranS}
%\bibliographystyle{IEEEtranSA}

% Here comes the bibliography/references.
% För att göra inställningar för IEEEtranS/SA kan man använda ett speciellt bibtex-entry @IEEEtranBSTCTL,
% se IEEEtran/bibtex/IEEEtran_bst_HOWTO.pdf, avsnitt VII, eller sista biten av IEEEtran/bibtex/IEEEexample.bib.
\bibliography{bibconfig,refs}
%\bibliography{refs}

\newpage
\appendix %%%% markerar att resten är appendixar
%%%% I er egen version, ta bort allt nedan (utom \end{document})
\input{text/appendix}

% Om ni har ett index
\makeatletter
\renewenvironment{theindex}
{\if@twocolumn
    \@restonecolfalse
  \else
    \@restonecoltrue
  \fi
  \twocolumn[\section{\indexname}]%
  \@mkboth{\MakeUppercase\indexname}%
  {\MakeUppercase\indexname}%
  \thispagestyle{plain}\parindent\z@
  \parskip\z@ \@plus .3\p@\relax
  \columnseprule \z@
  \columnsep 35\p@
  \let\item\@idxitem}
{\if@restonecol\onecolumn\else\clearpage\fi}
\makeatother
\printindex
\end{document}
