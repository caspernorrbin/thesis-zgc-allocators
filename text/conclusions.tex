This thesis has explored the adaptation and evaluation of different buddy allocators for integration within the Z Garbage Collector (ZGC). Through analysis and benchmarking, the binary tree allocator emerged as the most promising candidate, consistently outperforming the traditional binary buddy allocator in all key benchmarks. Although the iBuddy allocator demonstrated faster allocation speeds for smaller blocks, its poor performance with larger block allocations and significant internal fragmentation makes it impractical for uses where larger allocations occur regularly.

By tailoring the allocator to the specific needs of ZGC, several changes have been implemented to enhance efficiency and introduce additional features. Using already-available data allowed for further optimizations. Notably, a novel approach to initializing and employing the allocator was implemented: it begins fully occupied, with the user identifying which sections are empty and available for allocation. These modifications are designed to facilitate the seamless integration of the allocator into ZGC in the future.

To fully understand how the allocator would perform in ZGC, additional research is needed. So far, only proxy benchmarks have been utilized, showing promise for potential real-world applications. Therefore, future efforts should focus on developing strategies to mitigate the fundamental shortcomings of buddy allocators in terms of internal fragmentation and alignment. Additionally, integrating the allocator within ZGC would enable extensive testing and would be crucial in refining the allocator's performance.

%%% Local Variables:
%%% mode: latex
%%% TeX-master: "main"
%%% End:
