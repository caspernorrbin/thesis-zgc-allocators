Effective memory management is a crucial for any software system. It can be categorized into either manual memory management, where the developer is responsible for both allocating and deallocating memory, or automatic memory management, where the system handles memory on behalf of the developer. Garbage collection, a type of automatic memory management, identifies and reclaims memory that is no longer in use. There are various different implementations of garbage collection that achieve this goal.

Java applications run within a Java Virtual Machine (JVM), and one such example is the Open Java Development Kit (OpenJDK). OpenJDK includes various garbage collectors, including the Z garbage collector (ZGC). ZGC organizes memory into many pages, which are operated on concurrently during garbage collection. Objects are allocated sequentially on these pages through a method known as bump-pointer allocation, where a pointer tracks the position of the most recently allocated object, increasing it, or ``bumping'' it, with each new allocation.

Because bump-pointer allocation always starts from the end address of the last allocation, it is unable to reuse previously allocated memory on the same page, that has since become free. To resolve this, ZGC either relocates all active objects to a new page, allowing the original page to be reset, or moves all objects one by one to the top of the page. An alternative to this method is employing a free-list-based allocator, which maintains a list of all unoccupied memory, thereby allowing allocations in any available space, independent of prior allocations.

Using free-lists in garbage collectors is nothing new. Concurrent Mark Sweep (CMS) is a garbage collector that relies on free-lists. It was part of OpenJDK until its deprecation and subsequent removal. CMS could perform allocations without the constraints imposed by bump-pointer allocation methods, thanks to its use of free-lists. It is clear that free-lists offer certain advantages, but they may not always be the optimal choice for every situation. Using a free-list-based allocator within an existing garbage collector can offer significant advantages, as the garbage collector has more information available about the objects it allocates that the allocator can use. This thesis investigates the potential integration of a free-list-based allocator within ZGC, with focus on examining possible adaptations that can boost allocator efficiency within a garbage collection context, without actual implementation in ZGC. 

%%% Local Variables:
%%% mode: latex
%%% TeX-master: "main"
%%% End:
