In the current software development environment, Java remains one of the major languages, powering numerous applications. Central to Java's effectiveness is the Java Virtual Machine (JVM), with HotSpot being a key implementation. Within HotSpot, garbage collection (GC) is critical for efficient memory management, where one collector is Z (ZGC), designed for minimal latency and high throughput. \\

ZGC primarily uses bump-pointer allocation, which, while fast, can lead to fragmentation issues. An alternative allocation strategy involves using free-lists to dynamically manage memory blocks of various sizes, such as the buddy allocator. This thesis explores the adaptation and evaluation of buddy allocators for potential integration within ZGC, aiming to enhance memory allocation efficiency and minimize fragmentation. \\

The thesis investigates the binary buddy allocator, the binary tree buddy allocator, and the inverse buddy allocator, assessing their performance and suitability for ZGC. Although not integrated into ZGC, these exploratory modifications and evaluations provide insight into their behavior and performance in a GC context. The study reveals that while buddy allocators offer promising solutions to fragmentation, they require careful adaptation to handle the unique demands of ZGC. \\

The conclusions drawn from this research highlight the potential of free-list-based allocators to improve memory management in Java applications. These advances could reduce GC-induced latency and enhance the scalability of Java-based systems, addressing the growing demands of modern software applications.

%%% Local Variables:
%%% mode: latex
%%% TeX-master: "main"
%%% End:
